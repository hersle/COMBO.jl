%\documentclass[onecolumn]{aa} % for a paper on 1 column  
%\documentclass[longauth]{aa} % for the long lists of affiliations
%\documentclass[bibyear]{aa} % if the references are not structured according to the author-year natbib style
%\documentclass{aa}  
%\documentclass{memoir}
%\documentclass[oldfontcommands]{memoir}
%\documentclass[twocolumn]{revtex4-2}
%\documentclass[12pt,a4paper,oldfontcommands]{memoir}
\documentclass[10pt,a4paper]{article}
%\documentclass[10pt,a4paper,onecolumn]{paper}
\usepackage{fullpage}

\usepackage{amsmath}
\usepackage{amssymb}
\usepackage{graphicx}
%\usepackage{txfonts}
\usepackage[linktocpage=true, colorlinks=true, allcolors=blue]{hyperref}
\usepackage[labelfont=bf]{caption}
\usepackage[noabbrev]{cleveref}
\usepackage{tensor}
\usepackage{derivative}
\usepackage{booktabs}

\usepackage{biblatex}
%\bibliographystyle{plainnat}
\addbibresource{report.bib}

\newcommand\TODO[1]{\textcolor{red}{(\textbf{TODO:} #1)}}
\DeclareMathOperator{\asin}{asin}
\DeclareMathOperator{\sinc}{sinc}
\DeclareMathOperator{\diag}{diag}
%\setlength{\mathindent}{20pt}

\begin{document}

\title{\textbf{Solving the Einstein-Boltzmann equations}\\ \\\normalsize\textit{(AST5220 project report)}}
%\subtitle{Einstein-Boltzmann solver}
\author{Herman Sletmoen}

%\institute{ITA Oslo}
\date{Spring 2023}

\iffalse
\abstract
% context heading (optional)
{Solve Einstein-Boltzmann equations}
% aims heading (mandatory)
{Solve Einstein-Boltzmann equations}
% methods heading (mandatory)
{Solve Einstein-Boltzmann equations}
% results heading (mandatory)
{Solve Einstein-Boltzmann equations}
% conclusions heading (optional)
{Solve Einstein-Boltzmann equations}
\fi

%\keywords{cosmology -- Einstein-Boltzmann equations}

\maketitle
%
%________________________________________________________________

%\tableofcontents

\textbf{Code:} All code is available at \href{https://github.com/hersle/AST5220-project}{github.com/hersle/AST5220-project}.

\section{Introduction}

\TODO{WIP for now, eventually weave this together with the remaining milestones}

Einstein field equations:
\begin{equation}
	G\indices{_\mu_\nu} + \Lambda g\indices{_\mu_\nu} = \frac{8 \pi G}{c^4} T\indices{_\mu_\nu},
\label{eq_einstein}
\end{equation}
where $\Lambda$ is the cosmological constant.
Energy-momentum tensor:
\begin{equation}
	T\indices{^\mu_\nu} = ...
\label{eq_energy_momentum}
\end{equation}

Except in \cref{sec_supernova}, where we look for constraints on cosmological parameters, we use the Planck's cosmological parameters from 2018 \cite{planckcollaborationPlanck2018Results2020},
given and related by:
\begin{equation}
\begin{aligned}
	& \text{reduced Hubble parameter} & h &= 0.67, \\
	& \text{Hubble parameter} & H_0 &= h \cdot \frac{100\,\mathrm{km}}{\mathrm{s}\,\mathrm{Mpc}} = 67 \frac{\mathrm{km}}{\mathrm{s}\,\mathrm{Mpc}}, \\
	& \text{photon temperature} & T_{\gamma0} &= 2.7255\,K, \\
	& \text{effective neutrino number} & N_\text{eff} &= 3.046, \\
	& \text{matter density parameter} & \Omega_{b0} &= 0.05, \\
	& \text{cold dark matter density parameter} & \Omega_{c0} &= 0.267,\\
	& \text{curvature density parameter} & \Omega_{k0} &= 0, \\
	& \text{photon density parameter} & \Omega_{\gamma0} &= \frac{\pi^2}{15} \cdot \frac{(k_B T_{\gamma0})^4}{\hbar^3 c^5} \cdot \frac{8 \pi G}{3 H_0^2} = 0.000055, \\
	& \text{neutrino density parameter} & \Omega_{\nu0} &= N_{\rm eff}\cdot \frac{7}{8}\left(\frac{4}{11}\right)^{4/3}\Omega_{\gamma 0} = 0.000038, \\
	& \text{cosmological constant density parameter} & \Omega_{\Lambda 0} &= 1 - (\Omega_{k 0}+\Omega_{b 0}+\Omega_{c0}+\Omega_{\gamma 0}+\Omega_{\nu 0}) = 0.683,\\
	& \text{power spectrum spectral index} & n_s &= 0.965, \\
	& \text{power spectrum amplitude} & A_s &= 2.1\cdot 10^{-9}, \\
	& \text{\TODO{}} & Y_p &= 0.245, \\
	& \text{\TODO{}} & z_{\rm reion} &= 8, \\
	& \text{\TODO{}} & \Delta z_{\rm reion} &= 0.5, \\
	& \text{\TODO{}} & z_{\rm He reion} &= 3.5, \\
	& \text{\TODO{}} & \Delta z_{\rm He reion} &= 0.5.
\end{aligned}
\label{eq_planck2018}
\end{equation}


\clearpage

\section{Background cosmology}
\label{sec_background_cosmology}

According to the cosmological principle,
our Universe is spatially homogeneous and isotropic when averaged over large distances.
In this section, we will study the cosmology that describes such a universe
with radiation, matter, the cosmological constant and spatial curvature.
Formally, this is the zeroth-order ``background'' solution of the perturbed, inhomogeneous structured universe we aim to describe.

\subsection{Theory}
\label{sec_background_cosmology_theory}

\subsubsection{Friedmann equation}

The geometry of a \emph{spatially} homogeneous and isotropic universe
is described by the \textbf{Friedmann-Lemaître-Robertson-Walker (FLRW) metric}
\begin{equation}
\begin{split}
	ds^2 &= -c^2 dt^2 + a^2(t) \left[\frac{dr^2}{1 - kr^2} + r^2\left(d\theta^2 + \sin^2\theta \, d\phi^2\right) \right] \\
	     &= a^2(t) \left[-c^2 d\eta^2 + \frac{dr^2}{1 - kr^2} + r^2\left(d\theta^2 + \sin^2\theta \, d\phi^2\right) \right].
\end{split}
\label{eq_flrw}
\end{equation}
Here it is in spherical coordinates $(r,\theta,\phi)$, curvature $k$, scale factor $a(t)$,
and either cosmic time $t$ or conformal time $\eta$ defined by $d\eta = dt/a$\footnote{I prefer the convention in which conformal time is a time, while others like it to be the distance $d\eta \rightarrow c \, d\eta$.},
so it is conformal to the Minkowski metric in the flat case $k=0$.

The homogeneous background universe is filled with an ideal fluid that has a
time-dependent density $\rho(t)$, pressure $P(t)$ and thus energy-momentum tensor
\begin{equation}
	T\indices{^\mu_\nu} = \diag\Big[\rho(t),\, -P(t),\, -P(t),\, -P(t)\Big].
\label{eq_energy_momentum_homogeneous}
\end{equation}
The dynamics and evolution of the universe is governed by the Einstein field equations \eqref{eq_einstein},
with the FLRW metric \eqref{eq_flrw} on the left and energy-momentum tensor \eqref{eq_energy_momentum_homogeneous} on the right.
In addition to curvature and the cosmological constant,
we consider a universe with radiation and matter with densities $\rho_r$ and $\rho_m$,
and pressures given by their equations of state $P_m = 0$ and $P_r = \rho_r \, c^2 / 3$.
In particular, the $00$ and $11$-components give rise to the Friedmann equation for the Hubble parameter
\begin{equation}
	H(t) = \frac{1}{a} \odv{a}{t} = H_0 \sqrt{\Omega_{r0} \, a^{-4} + \Omega_{m0} \, a^{-3} + \Omega_{k0} \, a^{-2} + \Omega_{\Lambda0}},
\label{eq_friedmann}
\end{equation}
where radiation, matter, curvature and the cosmological constant have the (effective) mass densities
\begin{equation}
	\rho_r(t) = \rho_{r0} a^{-4}, \quad
	\rho_m(t) = \rho_{m0} a^{-3}, \quad
	\rho_k(t) = -\frac{3kc^2}{8 \pi G} a^{-2} = \rho_{k0} a^{-2}, \quad
	\rho_\Lambda(t) = \frac{\Lambda c^2}{8 \pi G} = \text{constant},
\end{equation}
and we define the time-dependent density parameters
\begin{equation}
	\Omega_s(t) = \frac{\rho_s(t)}{\rho_\text{crit}(t)}
\label{eq_density_parameters}
\end{equation}
relative to the critical density $\rho_\text{crit}(t) = 3H^2(t)/8\pi G$.
Present-day values are denoted by $F_0=F(t_0)$.

Due to the densities' differing dependencies on the scale factor,
a universe with all four species will be dominated by radiation early on, then matter, then curvature, and finally the cosmological constant.
We are mostly concerned with a flat universe where there is no curvature that can ever dominate,
so the scale factors at radiation-matter and matter-cosmological constant equality are found from
\begin{subequations}
\begin{align}
	\rho_r(t) &= \rho_m(t)       && \text{at} && a = a_\text{eq}^{rm} = \frac{\rho_{r0}}{\rho_{m0}} = \frac{\Omega_{r0}}{\Omega_{m0}}, \label{eq_equality_radiation_matter} \\
	\rho_m(t) &= \rho_\Lambda(t) && \text{at} && a = a_\text{eq}^{m\Lambda} = \left(\frac{\rho_{m0}}{\rho_{\Lambda}}\right)^\frac13 = \left(\frac{\Omega_{m0}}{\Omega_{\Lambda0}}\right)^\frac13. \label{eq_equality_matter_cosmological_constant}
\end{align}
\label{eq_equality_times}
\end{subequations}

Moreover, the acceleration of the scale factor
\begin{equation}
	\ddot{a} = \odv[2]{a}{t} = -a^2 H_0^2 \Big( 2 \, \Omega_{r0} a^{-4} + \Omega_{m0} a^{-3} - 2 \, \Omega_{\Lambda0} \Big).
\label{eq_acceleration}
\end{equation}
will be negative at early times as the expansion slows down,
but the expansion starts to accelerate as the cosmological constant takes over.

In addition to the ``cosmic'' Hubble parameter \eqref{eq_friedmann},
we define the \textbf{conformal Hubble parameter}
\begin{equation}
	\mathcal{H} = \frac1a \odv{a}{\eta} = \odv{a}{t} = a H.
\label{eq_conformal_hubble}
\end{equation}
For example, in universes dominated by radiation, matter and the cosmological constant,
we have,
\begin{subequations}
\begin{align}
	\mathcal{H} &= H_0 \sqrt{\Omega_{r0}} \, a^{-1} && \Big( \Omega_r \gg \{\Omega_m,\Omega_\Lambda\} \Big), \label{eq_conformal_hubble_dominated_radiation} \\
	\mathcal{H} &= H_0 \sqrt{\Omega_{m0}} \, a^{-1/2} && \Big( \Omega_m \gg \{\Omega_r,\Omega_\Lambda\} \Big), \\
	\mathcal{H} &= H_0 \sqrt{\Omega_{\Lambda0}} \, a && \Big( \Omega_\Lambda \gg \{\Omega_m,\Omega_r\} \Big).
\end{align}
\label{eq_conformal_hubble_dominated}
\end{subequations}

\subsubsection{Cosmic and conformal time}

In the form \eqref{eq_friedmann},
the Friedmann equation is a differential equation for the scale factor $a(t)$
as a function of the cosmic time $t$.
We will ``exchange'' $a(t) \leftrightarrow t(a)$, and instead parametrize the evolution of the universe with (the natural logarithm of) the scale factor $x = \log a$.
\textbf{This requires that $a(t)$ is monotonically increasing, so it is one-to-one with $t$!}
This \emph{always} holds in a universe with $\Omega_{k} \geq 0$,
but \textbf{breaks for some $\Omega_{k} < 0$, where the universe can have a turnaround $\dot{a} = 0$}.
We will always consider a flat universe and be fine,
except in \cref{sec_supernova} where we make a small detour and allow for nonzero curvature.

Parametrizing with $a$ or $x$, we can solve $H = \frac1a \odv{a}{t}$ and $d\eta = dt / a$ for the cosmic and conformal times
\begin{equation}
	t(a) = \int_0^a \frac{da}{aH} = \int_0^x \frac{dx}{H}
	\qquad \text{and} \qquad
	\eta(a) = \int_0^a \frac{da}{a^2 H} = \int_0^x \frac{dx}{aH}.
\label{eq_cosmic_conformal_time}
\end{equation}
In general, these integrals must be computed numerically.
However, in a flat universe with no cosmological constant and radiation-matter equality \eqref{eq_equality_radiation_matter},
they can be evaluated analytically to give
\begin{subequations}
\begin{align}
	t(a) &= \int_0^a \frac{da}{H_0 \sqrt{\Omega_{r0}a^{-2} + \Omega_{m0}a^{-1}}}
	      = \frac{2}{3 H_0 \sqrt{\Omega_{m0}}} \Big[\sqrt{a + a_\text{eq}} \big(a - 2 a_\text{eq}\big) + 2 a_\text{eq}^\frac32 \Big] && \Big(\Omega_k=\Omega_\Lambda=0\Big)
	\label{eq_cosmic_time_anal}, \\
	\eta(a) &= \int_0^a \frac{da}{H_0 \sqrt{\Omega_{r0} + \Omega_{m0} a}}
		     = \frac{2}{H_0 \sqrt{\Omega_{m0}}} \Big[ \sqrt{a + a_\text{eq}} - \sqrt{a_\text{eq}}\Big] && \Big(\Omega_k=\Omega_\Lambda=0\Big)
	\label{eq_conformal_time_anal}.
\end{align}
\label{eq_cosmic_conformal_time_anal}
\end{subequations}
In a universe with only radiation, these expressions reduce further to
\begin{align}
	t &= \frac{a^2}{2 H_0 \sqrt{\Omega_{r0}}}
	\qquad \text{and} \qquad
	\eta = \frac{a}{H_0 \sqrt{\Omega_{r0}}}
	&& \Big(\Omega_m=\Omega_k=\Omega_\Lambda=0\Big).
\label{eq_cosmic_conformal_time_anal_radiation}
\end{align}

To interpret these times physically, we glance back at the FLRW metric \eqref{eq_flrw}.
First, we see that cosmic time $t$ is the proper time of \emph{fundamental observers} that move with the expansion,
so a clock with zero peculiar velocity ticks at a rate corresponding to the cosmic time.
Second, we see that photons travel infinitesimal distances $c\,d\eta$ on the comoving grid (in $[\ldots]$),
so we can interpret the \textbf{(comoving) horizon}
\begin{equation}
D_\text{hor} = c \, \eta = c \int_0^\eta d\eta = c \int_0^t \frac{dt}{a(t)} = c \int_0^a \frac{da}{a^2 H} = c \int_0^x \frac{dx}{aH}
\label{eq_horizon}
\end{equation}
as the maximum (comoving) distance within which there can be causal communication.
This quantity will be relevant in \TODO{ref later milestone}.


\subsection{Implementation}
\label{sec_background_cosmology_implementation}

\begin{itemize}
	\item We represent a $\Lambda$CDM cosmology with an object that takes $h$, $\Omega_{b0}$, $\Omega_{c0}$, $\Omega_{k0}$, $T_{\gamma0}$ and $N_\text{eff}$ as free parameters,
	      and then computes the remaining dependent parameters from the relations \eqref{eq_planck2018}.
	\item We check whether a universe has a turnaround $\dot{a} = 0$ and our parametrization with $x$ breaks down.
	      This comes to use in \cref{sec_supernova}, where we allow for arbitrary curvature.
	\item We compute $\mathcal{H}(x)$ and its derivatives $\odv{\mathcal{H}}/{x}$ and $\odv[2]{\mathcal{H}}/{x}$ analytically based on the Friedmann equation \eqref{eq_friedmann}.
	      The derivatives can be derived without a lot of work by noting that
		  $E(x) = \Omega_{r0} e^{-4x} + \Omega_{m0} e^{-3x} + \Omega_{k0} e^{-2x} + \Omega_{\Lambda0}$
		  in the square root of the Hubble parameter has the derivatives
		  \begin{equation*}
			  \odv[d]{E}{x} = (-4)^d \Omega_{r0} a^{-4}(x) + (-3)^d \Omega_{m0} a^{-3}(x) + (-2)^d \Omega_{k0} a^{-2}(x) \qquad \big(d \geq 1\big).
		  \end{equation*}
	\item We compute the density parameters \eqref{eq_density_parameters} from today's density parameters using, for example for radiation, $\Omega_r = \rho_r / \rho_\text{crit} = (\rho_{r0} a^{-4} / \rho_{\text{crit},0}) (\rho_{\text{crit},0} / \rho_\text{crit}) = \Omega_{r0} a^{-4} (H_0 / H)^2$.
	\item We compute the radiation-matter and matter-cosmological constant equalities \eqref{eq_equality_times} analytically,
	      but the onset of the acceleration \eqref{eq_acceleration} numerically from when $\odv{\mathcal{H}}/{x} = \ddot{a} / H = 0$.
	\item We compute the cosmic time $t(x)$ and conformal time $\eta(x)$ by inserting their derivatives $\odv{t}/{x}$ and $\odv{\eta}/{x}$ into an adaptive 4th(5th)-order Runge-Kutta integrator.
	      As it is computationally infeasible to integrate from $x=-\infty$,
	      we start from a small initial value, like $x = -20$,
	      and the corresponding analytical cosmic or conformal time \eqref{eq_cosmic_conformal_time_anal} in a universe dominated by radiation and matter.
	\item We store the integrated $t(x)$ and $\eta(x)$ on a cubic spline, so subsequent evaluations are fast.
\end{itemize}

\subsection{Tests and results}

We create a cosmology with the Planck 2018 parameters \eqref{eq_planck2018} and study its evolution.

\begin{figure}[b!]
	\centering
	\includegraphics[scale=0.7]{../plots/density_parameters.pdf}
\caption{Evolution of the density parameters \eqref{eq_density_parameters} in the Planck 2018 cosmology \eqref{eq_planck2018}.}
\label{fig_density_parameters}
\end{figure}

\Cref{fig_density_parameters} shows that the universe
transitions from being dominated by radiation to matter to the cosmological constant,
with the equality times \eqref{eq_equality_times} reported in \cref{table_times}.
In this cosmology, there is no curvature $\Omega_{k} = \Omega_{k0} = 0$,
and all density parameters sum to $\Omega_{r} + \Omega_m + \Omega_k + \Omega_\Lambda = 1$ at all times -- as they should, by the Friedmann equation \eqref{eq_friedmann} and definition \eqref{eq_density_parameters}.

\begin{figure}
	\centering
	\includegraphics[scale=0.7]{../plots/conformal_hubble.pdf}
	\includegraphics[scale=0.7]{../plots/conformal_hubble_derivative1.pdf}
	\includegraphics[scale=0.7]{../plots/conformal_hubble_derivative2.pdf}
	\caption{%
		Evolution of the conformal Hubble parameter \eqref{eq_conformal_hubble} and its two derivatives in the Planck 2018 cosmology \eqref{eq_planck2018},
		compared to their values from the analytical expression \eqref{eq_conformal_hubble_dominated} in dominated universes.
		Dashed lines show the equality times from \cref{fig_density_parameters},
		while the dotted line indicates the onset of the acceleration \eqref{eq_acceleration}.
	}
	\label{fig_conformal_hubble}
\end{figure}

\Cref{fig_conformal_hubble} shows the evolution of the conformal Hubble parameter \eqref{eq_conformal_hubble} and its two derivatives.
Note that the expansion rate $\dot{a}$ decreases most quickly during radiation domination and slower during matter domination,
but the universe starts to \emph{accelerate} slightly before $\Omega_m = \Omega_\Lambda$, at the time reported in \cref{table_times}.
This is caused by the rise of the cosmological constant, and its effective negative pressure.
Moreover, during the three dominated eras,
the evolution is consistent with the analytical expectation \eqref{eq_conformal_hubble_dominated}.

\begin{figure}
	\centering
	\includegraphics[scale=0.7]{../plots/times.pdf}
	\caption{%
		Evolution of numerically integrated cosmic and conformal times \eqref{eq_cosmic_conformal_time} in the Planck 2018 cosmology \eqref{eq_planck2018},
		compared to the analytical expressions \eqref{eq_cosmic_conformal_time_anal} in a universe with no cosmological constant.
	}
	\label{fig_cosmic_conformal_time}

	\bigskip

	\includegraphics[scale=0.7]{../plots/eta_H.pdf}
	\caption{%
		Evolution of the product between the conformal time \eqref{eq_cosmic_conformal_time} and conformal Hubble parameter \eqref{eq_conformal_hubble},
		compared to that with the analytical time \eqref{eq_conformal_time_anal} and the Hubble parameter with $\Omega_{k0}=\Omega_{\Lambda0}=0$.
	}
	\label{fig_eta_H}
\end{figure}

\Cref{fig_cosmic_conformal_time} shows the relation between the scale factor and cosmic and conformal time \eqref{eq_cosmic_conformal_time} from numerical integration.
Before the cosmological constant becomes important, they closely match the analytical times \eqref{eq_cosmic_conformal_time_anal} from a universe with only radiation and matter.
We can also read off the current age of the universe, as reported in \cref{table_times}.

\Cref{fig_eta_H} shows the evolution of the product $\eta \mathcal{H}$.
The former plots indicate that our computation of conformal time and the Hubble parameter work independently,
and this shows that so does the combination.
Through radiation-domination and matter-domination,
it follows the value we expect from the analytical expression \eqref{eq_conformal_time_anal}
and the Hubble parameter with $\Omega_{k} = \Omega_\Lambda = 0$.
In particular, as $x \rightarrow -\infty$ and radiation dominates,
the product between the conformal time \eqref{eq_cosmic_conformal_time_anal_radiation}
and the conformal Hubble parameter \eqref{eq_conformal_hubble_dominated_radiation} converges to 1.

\begin{table}
\centering
\caption{%
	The time of occurence of four important events in the evolution of a universe with the Planck cosmology \eqref{eq_planck2018},
	expressed in terms of the scale factor $a$, its natural logarithm $x = \log a$, redshift $z = \frac1a - 1$, cosmic time $t$ and conformal time $\eta$.
}
\label{table_times}
\begin{tabular}{l c c c c c}
	\toprule
	Event                                                               & $x$     & $a$       & $z$    & $\eta$    & $t$ \\
	\midrule
	Radiation-matter equality ($\Omega_r = \Omega_m$)                   & $-8.13$ & $0.0003$  & $3400$ & $0.4\,\mathrm{Gyr}$ & $50\,\mathrm{kyr}$ \\
	Acceleration onset ($\ddot{a} = 0$)                                 & $-0.49$ & $0.61$    & $0.63$ & $38.5\,\mathrm{Gyr}$ & $7.8\,\mathrm{Gyr}$   \\
	Matter-cosmological constant equality ($\Omega_m = \Omega_\Lambda$) & $-0.26$ & $0.77$    & $0.29$ & $42.3\,\mathrm{Gyr}$ & $10.4\,\mathrm{Gyr}$  \\
	Today ($t = t_0$)                                                   & $0$     & $1$       & $0$    & $46.3\,\mathrm{Gyr}$ & $13.8\,\mathrm{Gyr}$  \\
	\bottomrule
\end{tabular}
\end{table}

\clearpage

\section{Cosmological constraints from supernovae}
\label{sec_supernova}

In this section, we forget most of the Planck cosmological parameters \eqref{eq_planck2018} for a moment;
neglecting neutrinos by fixing $N_\text{eff}=0$ and keeping only $T_{\gamma0}$, hence fixing $\Omega_{r0}$.
Instead, we constrain the independent parameters $h$, $\Omega_{m0}$ and $\Omega_{k0}$,
and hence the dependent $\Omega_{\Lambda 0}=1-\Omega_{k0}-\Omega_{m0}-\Omega_{r0}$,
using observed supernovae luminosity distances from \cite{betouleImprovedCosmologicalConstraints2014}.
To do so, we do a Markov chain Monte Carlo (MCMC) analysis
by stepping through cosmologies with various parameters using the Metropolis-Hastings algorithm
and comparing their predicted luminosity distances to the data.

\subsection{Theory}

\subsubsection{Cosmological distances}

From the FLRW metric \eqref{eq_flrw} and conformal time \eqref{eq_cosmic_conformal_time},
we can show how to compute distances in the universe.
Consider a photon traveling on a radial path with $d\theta = d\phi = 0$,
from emission at $(\eta,r)$ to our observation at $(\eta_0, 0)$,
along the null geodesic
\begin{equation*}
	0 = ds^2 = a^2(t) \left[ -c^2 d\eta^2 + \frac{dr^2}{1-kr^2} \right].
\end{equation*}
On the comoving grid (in $[\ldots]$), it travels the \textbf{comoving distance}
\begin{equation}
	\chi = \int_{\eta}^{\eta_0} c \, d\eta = c \, \big(\eta_0 - \eta\big) = \int_r^0 \frac{-dr}{\sqrt{1-kr^2}} = \frac{\asin\big(\sqrt{k}r\big)}{\sqrt{k}},
\label{eq_comoving_distance}
\end{equation}
so it came from the radial coordinate%
\footnote{This holds for all $k$ as $\sinc(x) = \sin x / x$ takes complex arguments, with $\sin(ix) = i \sinh x$ and $\sinc(0) = 1$.}
\begin{equation}
	r = \frac{\sin\Big(\sqrt{k}\chi\Big)}{\sqrt{k}} = \chi \sinc\Big(\sqrt{k}\chi\Big).
\label{eq_radial_coordinate}
\end{equation}

Given the observed redshift $z$ of light,
we can then compute its scale factor $a = (z+1)^{-1}$ at emission,
the corresponding conformal time \eqref{eq_cosmic_conformal_time},
the comoving distance \eqref{eq_comoving_distance}, the radial coordinate \eqref{eq_radial_coordinate}
and thus the corresponding \textbf{angular diameter distance} and \textbf{luminosity distance}
\begin{equation}
	d_A = a r
	\qquad \text{and} \qquad
	d_L = \frac{r}{a} = \frac{d_A}{a^2}.
\label{eq_distances}
\end{equation}

\subsubsection{Statistics}

From \cite{betouleImprovedCosmologicalConstraints2014},
we have measured luminosity distances $d_{L}^\text{obs}(z_i)$ and their
corresponding measurement uncertainties $\sigma_i^\text{obs}$
for $N=31$ different redshifts $z_i$.
Given the three cosmological parameters $h$, $\Omega_{m0}$ and $\Omega_{k0}$,
we can then fit the data to corresponding theoretically predicted distances $d_L(z_i; h, \Omega_{m0}, \Omega_{k0})$.
Assuming the different measurements are Gaussian distributed and uncorrelated,
the likelihood function that rates the fit is $L \propto e^{-\chi^2/2}$, where the $\chi^2$-function is
\begin{equation}
	\chi^2(h,\Omega_{m0},\Omega_{k0}) = \sum_{i=1}^{N} \left( \frac{d_L(z_i; h, \Omega_{m0}, \Omega_{k0}) - d_{L}^\text{obs}(z_i)}{\sigma_i^\text{obs}} \right)^2.
\label{eq_chi2}
\end{equation}

The Metropolis-Hastings algorithm steps through various combinations of $\mathbf{p} = (h,\Omega_{m0},\Omega_{k0})$ in parameter space, measuring their likelihood $L(\mathbf{p})$.
Each iteration $i$, it randomly shifts the parameters from their current values with a normal distribution,
and then records their new values as a random sample of their probability distribution with probability $\min\big\{L_{i+1}/L_i, 100\%\big\}$.

By the central limit theorem, once the algorithm has gathered many samples $\mathbf{p}_i$,
they should scatter around the \emph{best fit}
with maximum $L(\mathbf{p}_\text{best}) = \max\{L(\mathbf{p}_i)\}$
like a multivariate Gaussian with the same dimension $D$ as the parameter space.
We can then produce \emph{confidence regions} for the parameters
by identifying contours that enclose a given fraction $F$ of the samples.
For a multivariate Gaussian distribution, a fraction $F$ is enclosed by an ellipsoid
for which
\begin{equation}
	\chi^2_i - \chi^2_\text{best} < q_{\chi^2_D}(F),
\label{eq_confidence_region}
\end{equation}
where $q_{\chi^2_D}(F)$ is the inverse cumulative distribution function of the $\chi^2$-distribution with $D$ degrees of freedom.
We have $D=3$ independent parameters, and look for standard $68.3\%$ and $95.4\%$ confidence regions
with $q_{\chi^2_3}(68.3\%) \approx 3.53$ and $q_{\chi^2_3}(95.4\%) \approx 8.00$.

\subsection{Implementation}

\begin{itemize}
	\item We roll our own homemade Metropolis-Hastings algorithm.
	      It takes a function that computes the likelihood $L(\mathbf{p})$ for a set of parameters $\mathbf{p}$.
	      Unless specified explicitly, it sets step sizes of the parameters as a proportion of their lower and upper bounds,
		  and adaptively scales them if the algorithm accepts samples at a rate too far from the ``optimal'' acceptance rate around $25\%$ \cite{gelmanWeakConvergenceOptimal1997}.
		  The algorithm can run multiple chains from different initial parameter guesses,
		  each with a requested number of (accepted) samples after removing a given number of burn-in samples.
	\item We exclude parameters outside their specified bounds by assigning $L=0$ to them.
	\item As mentioned in \cref{sec_background_cosmology_theory},
	      our implementation of the background cosmology parametrized by the scale factor
	      \textbf{cannot handle cosmologies with turnaround} $\dot{a} = 0$.
	      These cosmologies can arise now that we allow $\Omega_{k0} \neq 0$,
	      for example with $\Omega_{r0}=0$, $\Omega_{m0} = 0.2$ and $\Omega_{k0} = -0.9$ and $\Omega_{\Lambda0} = 1.7$.
	      We identify such cosmologies as described in \cref{sec_background_cosmology_implementation}, and exclude them by setting $L=0$.
\end{itemize}

\subsection{Results}

\begin{figure}[!b]
	\centering
	\includegraphics[scale=0.7]{../plots/supernova_distance.pdf}
	\caption{Observed and predicted luminosity distances \eqref{eq_distances} from \cite{betouleImprovedCosmologicalConstraints2014} and the Planck cosmology \eqref{eq_planck2018}. \TODO{\textbf{bad} agreement! discuss! add best fit!}}
	\label{fig_luminosity_distances}
\end{figure}

\Cref{fig_luminosity_distances} shows observed and predicted luminosity distances from the Planck 2018 cosmology \eqref{eq_planck2018}.
The agreement is already relatively good, so we expect that our constraints should be close to their values.

\begin{figure}[b]
	\centering
	\includegraphics[scale=0.7]{../plots/supernova_hubble.pdf}
	\includegraphics[scale=0.7]{../plots/supernova_omegas.pdf}
	\caption{%
		Probability distribution of today's reduced Hubble parameter $h$,
		and confidence regions \eqref{eq_confidence_region} for $\Omega_{m0}$ and $\Omega_{\Lambda0}$,
		from $10 \times 10000$ Metropolis-Hastings samples with $L \propto e^{-\chi^2/2}$ and the $\chi^2$ sum \eqref{eq_chi2},
		comparing predicted luminosity distances \eqref{eq_distances} to observations from \cite{betouleImprovedCosmologicalConstraints2014}.
		The algorithm restricts the parameters to the prior bounds $h \in [0.5, 1.5]$, $\Omega_{m0} \in [0, 1]$ and $\Omega_{k0} \in [-1, +1]$, and accordingly (with negligible radiation today) $\Omega_{\Lambda0} \in [-1, 2]$.
	}
	\label{fig_supernova_mcmc}
\end{figure}

\Cref{fig_supernova_mcmc} shows our MCMC constraints on $h$, $\Omega_{m0}$ and $\Omega_\Lambda$,
from the prior bounds $h \in [0.5, 1.5]$, $\Omega_{m0} \in [0, 1]$ and $\Omega_{k0} \in [-1, +1]$
that accommodate a wide region around the Planck values \eqref{eq_planck2018}, for example.
In addition, curved universes with a turnaround $\dot{a} = 0$ are forbidden
(\cite[Figure 11]{amanullahSpectraLightCurves2010} shows that such cosmologies are disconnected from the best fit regions).

Note that the constraint in the $\Omega_{m0}$-$\Omega_{\Lambda0}$-plane is highly orthogonal to the line of flat universes,
so supernova data can give good constraints when combined with some other constraint that argues in favor of flatness, for example.
Our best fits for $\Omega_{m0}$ and $\Omega_{\Lambda0}$ agrees relatively well with both Planck's and a similar analysis in \cite[Fig. 15]{betouleImprovedCosmologicalConstraints2014}.
On the other hand, our Hubble parameter is somewhat larger than Planck's and exemplifies the Hubble tension.

\TODO{discuss bad Planck 2018 predictions}

\TODO{maybe also quoting the constraints for all parameters in the text}

\clearpage
\section{Recombination and reionization history}

Photons (of the CMB) observed today were emitted at various times in the past.
In this section, our main goal is to accurately compute the optical depth and visibility function of the universe,
which play key roles in predicting the today's observed CMB spectrum.
The most important process that affects this is Thomson scattering
\begin{equation}
	e^- + \gamma \leftrightarrow e^- + \gamma
\label{eq_thomson_scattering}
\end{equation}
of photons off free electrons,
so to do so we have to study the evolution of free electrons through the recombination and reionization history of the universe.

\subsection{Theory}

Light emitted from a source with intensity $I_0$,
traveling through a medium with \textbf{optical depth} $\tau$ that scatters certain photons,
is observed at the other end with intensity $I = I_0 e^{-\tau}$.
In a cosmological context, we take this medium to be a gas of electrons with number density $n_e$
and the main scattering mechanism to be Thomson scattering with cross-section $\sigma_T = (8\pi/3) (\alpha \hbar / m_e c)^2$.
In this case, the optical depth of photons emitted at conformal time $\eta$ and observed today at $\eta_0$ is
\begin{equation}
	\tau(\eta) = \int_\eta^{\eta_0} n_e \sigma_T a c \, d\eta = \int_x^0 \frac{n_e \sigma_T c}{H} \, dx.
\label{eq_optical_depth}
\end{equation}

A natural quantity derived from the optical depth is the \textbf{visibility function}
\footnote{here wrt $x$, other places wrt another time variable $\eta$. whichever definition gives the interpretation that it is the probability ... at some time $x$, with integral $1$.}
\begin{equation}
	g(\eta) = -\tau'(\eta) \, e^{-\tau(\eta)} = \Big( e^{-\tau(\eta)} \Big)'
	\qquad \text{or} \qquad
	\tilde{g}(x) = -\tau'(x) \, e^{-\tau(x)} = \Big( e^{-\tau(x)} \Big)',
\label{eq_visibility_function}
\end{equation}
here defined in terms of the two different time variables $\eta$ and $x$.
As $\tau=0$ today and $\tau \rightarrow \infty$ at early times,
we see that they both have the property $\int_0^{\eta_0} g(\eta) d\eta = \int_{-\infty}^0 \tilde{g}(x) dx = 1$.
They can therefore be interpreted as probability densities for a photon observed today being last scattered off a free electron at time $\eta$ or $x$.

We already know how all the quantities in the optical depth \eqref{eq_optical_depth} evolves, except the free electron density $n_e$.
Therefore, we set out to compute the free electron density $X_e = n_e / n_H$,
where $n_H \approx (1-Y_p) \rho_b / m_H$ is the number density of Hydrogen,
$Y_p$ is the primordial mass fraction of Helium
and $1-Y_p$ that of Hydrogen.

Electrons (re)combine with protons and ionized Helium ($\rightarrow$),
and photons can (re)ionize Hydrogen and Helium $(\leftarrow$):
\begin{subequations}
\begin{align}
	e^- + \text{H}^+ &\leftrightarrow \text{H} + \gamma,\\
	e^- + \text{He}^+ &\leftrightarrow \text{He} + \gamma, \\
	e^- + \text{He}^{++} &\leftrightarrow \text{He}^+ + \gamma.
\end{align}
\label{eq_thomson}
\end{subequations}
Before recombination, when these processes are (almost) in equilibrium,
this gives rise to the three \textbf{Saha equations} (for H+He)
\TODO{charge neutral}
\TODO{reorder?}
\newcommand{\XHp}{X_{\text{H}^+}}
\newcommand{\XHep}{X_{\text{He}^+}}
\newcommand{\XHepp}{X_{\text{He}^{++}}}
\begin{subequations}
\begin{align}
    \frac{\XHepp}{1 - \XHep - \XHepp} &= 2 \, \frac{\lambda_e^{-3}}{n_e} \exp \left(-\frac{E^\text{ion}_\text{He1}}{k_B T_b} \right), \\
    \frac{\XHepp}{\XHep} &= 4 \, \frac{\lambda_e^{-3}}{n_e} \exp\left(-\frac{E^\text{ion}_\text{He2}}{k_B T_b}\right), \\
    \frac{\XHp}{1-\XHp} &= \phantom{1} \, \frac{\lambda_e^{-3}}{n_e} \exp\left(-\frac{E^\text{ion}_\text{H}}{k_B T_b}\right), \\
\end{align}
\label{eq_saha_H_He}
\end{subequations}
\TODO{ionization energies in $\leftrightarrow$ reactions above}
\TODO{define $X$s}
where $\lambda_e = h / \sqrt{2 \pi m_e k_B T_b}$ is the de Broglie wavelength of electrons with the baryon temperature $T_b$,
for the free electron fraction
\begin{equation}
    X_e = \XHp + \frac{Y_p}{4(1-Y_p)} \Big( \XHep + 2 \XHepp \Big) .
\label{eq_Saha_H_He_Xe}
\end{equation}
Without Helium ($Y_p=0$) or when Helium has recombined ($\XHp \gg \XHep \gg \XHepp$),
this reduces to the Hydrogen-only Saha equation
\begin{equation}
	\frac{X_e^2}{1-X_e} = \frac{\lambda_e^{-3}}{n_b} \exp \left( -\frac{E^\text{ion}_\text{H}}{k_B T_b} \right).
\label{eq_saha_H}
\end{equation}

During and after recombination, the Saha equation is no longer valid,
and the evolution of $X_e$ is more accurately described by the \textbf{Peebles equation}
\TODO{summarize difference from Saha}
\begin{equation}
	\odv{X_e}{x} = \frac{C_r}{H} \left( \beta (1-X_e) - n_H \alpha_2 X_e^2 \right),
\label{eq_Peebles}
\end{equation}
with the many easy-to-compute quantities
\begin{subequations}
\begin{align}
	C_r &= \frac{\Lambda_{2s \rightarrow 1s} + \Lambda_\alpha}{\Lambda_{2s \rightarrow 1s} + \Lambda_\alpha + \beta_2}, \\
	\alpha &= \text{fine structure constant}, \\
	\alpha_2 &= \frac{64 \pi}{\sqrt{27 \pi}} \left(\frac{\alpha}{m_e}\right)^2 \sqrt{\frac{E^\text{ion}_\text{H}}{k_B T_b}} \phi_2 \hbar^2/c, \\
	\beta &= \frac{\alpha_2}{\lambda_e^3} \exp \left(-\frac{E^\text{ion}_\text{H}}{k_B T_b}\right), \\
	\beta_2 &= \beta \exp \left(\frac{3 E^\text{ion}_\text{H}}{4 k_B T_b}\right), \\
	\Lambda_{2s \rightarrow 1s} &= 8.227 / \mathrm{s}, \\
	\Lambda_\alpha &= H \frac{(3 E^\text{ion}_{\text{H}})^3}{(\hbar c)^3 (8 \pi)^2 n_{1s}}, \\
	\phi_2 &= 0.448 \log \left( \frac{E^\text{ion}_\text{H}}{k_B T_b} \right), \\
	n_{1s} &= (1-X_e) n_H.
\end{align}
\end{subequations}

\TODO{is it only ``difficult to integrate'' Peebles at early times, or is it \emph{wrong} there?}

The full evolution of the free electron fraction is therefore found by stitching together the predictions of the Saha and Peebles equation
\begin{equation}
	X_e(x) = \begin{cases}
	             X_e^\text{Saha}(x) & \text{(for $X_e^\text{Saha} > 0.999$)}, \\
	             X_e^\text{Peebles}(x) & \text{(for $X_e^\text{Saha} < 0.999$)}, \\
	         \end{cases}
\label{eq_free_electron_fraction}
\end{equation}
where $X_e^\text{Saha}$ can be found analytically or through fixed-point iteration,
while $X_e^\text{Peebles}$ must be found by numerically integrating equation \eqref{eq_Peebles}
\emph{from the time where $X_e^\text{Saha} = 0.999$} (see implementation details).

We also include a phenomenological reionization model,
where we ramp up the free electron fraction according to
\TODO{meaning of $f$?}
\TODO{$y$ or $z$? in last term?}
\begin{equation}
	X_e \rightarrow X_e + \frac{1+f_\text{He}}{2} \left[ 1 + \tanh \left( \frac{y^\text{reion}_1-y}{\Delta y^\text{reion}_1} \right) \right]
                        + \frac{f_\text{He}}{2} \left[ 1 + \tanh \left( \frac{y^\text{reion}_1-2}{\Delta y^\text{reion}_2} \right) \right]
\label{eq_reionization}
\end{equation}
where $y = (1+z)^{3/2}$, $\Delta y = |y'(z)| \Delta z$ and we have the four reionization parameters in \ref{eq_planck2018}.

In the above expressions, we need the baryon temperature $T_b(t)$.
While the photon temperature follows the simple black-body evolution
\begin{equation}
	T_\gamma(t) = \frac{T_{\gamma 0}}{a(t)},
\label{eq_photon_temperature}
\end{equation}
At early times, the baryons and photons are coupled and share the common temperature
\begin{equation}
	T_b(t) = T_{\gamma}(t).
\label{eq_baryon_temperature}
\end{equation}
After recombination and decoupling, however,
the baryon temperature generally evolves according to a differential equation that is coupled to the Peebles equation.
However, for our purposes of computing $X_e$,
the error in assuming the common temperature \eqref{eq_baryon_temperature} is only of order $10^{-6}$ \TODO{ref},
so we use this approximation through all time.

The last quantity we introduce is the sound horizon
of the coupled photon-baryon plasma.
Before recombination, its sound speed is
$c_s(\eta) = c \sqrt{R(a) / 3(1+R(a))}$,
where $R(a) = 4 \Omega_{\gamma 0} / 3 \Omega_{b 0} a$,
so the \textbf{sound horizon} is
\begin{equation}
	s(\eta) = \int_0^\eta c_s(\eta) d\eta = \int_{-\infty}^x \frac{c_s \, dx}{\mathcal{H}}.
\label{eq_sound_horizon}
\end{equation}

\subsection{Implementation}

\begin{itemize}
\item
We solve the H-only Saha equation \eqref{eq_saha_H}
with right side $R$
as the quadratic equation $X_e^2 + R X_e - R = 0$.
Its positive solution is $X_e = (-R + \sqrt{R^2+4R})/2$,
but since $R$ can skyrocket and overflow,
we take the second-order Taylor expansion $X_e = (R/2) (1 - \sqrt{1 + 4/R}) \simeq 1 - 1/R$
for $R > 10^{10}$ that can only underflow, avoiding numerical issues.

\item
We solve the H+He Saha equation \eqref{eq_saha_H_He} with fixed-point iteration.
Starting with an initial guess $X_e^{(i)}$,
we analytically compute the corresponding $n_e = n_H X_e$,
then $\XHp$, $\XHep$ and $\XHepp$ from equation \eqref{eq_saha_H_He},
and finally a new free electron fraction $X_e^{(i+1)}$ from equation \eqref{eq_Saha_H_He_Xe};
repeating the process until $|X_e^{(i+1)}-X_e^{(i)}| < 10^{-15}$.
We choose the initial guess $X_e^{(0)}$ as the fast H-only solution from above,
resulting in fewer iterations compared to using a constant initial guess, like $X_e^{(0)} = 1$.
Note that while few iterations are needed around $X_e \approx 1$,
the convergence becomes extremely slow as $X_e \rightarrow 0$.
Fortunately, in this regime, almost all Helium has already combined,
so we have $\XHepp \ll \XHep \ll \XHp$ and simply take the H-only solution when it is $X_e < 0.5$.

\item
We solve the Peebles equation with an ODE integrator and spline the result, like in \cref{sec_background_cosmology}.
As explained below equation \eqref{eq_free_electron_fraction},
we specify the initial values to the integrator
as $X_e = 0.999$ and the corresponding time $x$ found by a numerical root finder.
Note that the exponential in $\beta_2$ is unsafe and can overflow,
but substituting $\beta$ yields a new safe exponential that can only underflow.

\item
To compute $X_e(x)$ at any time $x$ with reionization,
we choose the Saha or Peebles equation from the branching \eqref{eq_free_electron_fraction},
and then add the reionization contribution \eqref{eq_reionization}.

\item
To compute the optical depth \eqref{eq_optical_depth} and the sound horizon \eqref{eq_sound_horizon},
we again use an ODE integrator and spline the result.
The former is integrated backwards from $\tau(\eta_0)=0$,
but the latter cannot start at $x_0 = -\infty$,
so we start it at the finite early time $x_0 = -20$ with $s(x_0) \approx c_s(x_0) / \mathcal{H}(x_0)$.

\item
We compute derivatives of splined quantities with spline-based methods,
but use analytical expressions for other quantities.
\end{itemize}

\subsection{Results}

\TODO{report table with times}

\TODO{report sound horizon. plot?}

Here we examine our results for the Planck 2018 cosmology \eqref{eq_planck2018},
with the reionization model turned on and off,
and with Hydrogen only or Hydrogen and Helium.

\begin{figure}
	\centering
	\includegraphics[scale=0.7]{../plots/free_electron_fraction_log.pdf}
	\includegraphics[scale=0.7]{../plots/free_electron_fraction_linear.pdf}
	\caption{%
		Evolution of the free electron fraction $X_e(x)$,
		from the Saha and Peebles equations before and after the joint $X_e = 0.999$,
		with and without Helium and reionization, compared to the result from the Saha equation alone.
		Unlike the logarithmic plot, the linear plot is annotated with the majority atom/ion at each recombination stage.
	}
	\label{fig_free_electron_fraction}
\end{figure}

\Cref{fig_free_electron_fraction} shows how the free electron fraction $X_e$ evolves through recombination and reionization.
As expected from \TODO{equation ref},
$X_e$ gradually steps down, plateau by plateau, as the recombinations $\text{He}^{++} \rightarrow \text{He}^+ \rightarrow \text{He}$ and $\text{H}^+ \rightarrow \text{H}$ occur sequentially,
as the thermal energy $k_B T_b$ decreases and reaches the different ionization energies at different times.
With reionization, $X_e$ ramps back up at late times because increasing radiation from formed stars force the electrons out of the neutral atoms,
but skipping the $\text{H}^+, \text{He}$ stage (\TODO{why?}).
%$X_e$ gradually steps down from plateaus where $\{\XHp, \XHep, \XHepp\}$ (almost) changes
%from $\{1,0,1\}$ to $\{1,1,0\}$ to $\{1,0,0\}$ to $\{0,0,0\}$

Note that without reionization, the Peebles equation would cause the free electron density to asymptotically freeze out today at $X_e(0) = 10^{\{-3.6, -3.7\}}$ with and without Helium.
In contrast, the Saha equation dives straight to $0$ during the final recombination to Hydrogen.

\begin{figure}
	\centering
	\includegraphics[scale=0.7]{../plots/optical_depth.pdf}
	\caption{\TODO{$\tau'(\eta(x))$} Evolution of the optical depth $\tau(x)$ and its derivatives through recombination and reionization history, with or without both Helium and reionization.}
	\label{fig_optical_depth}
\end{figure}

\Cref{fig_optical_depth} shows the corresponding evolution of the optical depth $\tau$.
The optical depth measures the ``thickness'' of what the most distant light (from the horizon \TODO{?}) reaching us today has traveled through,
so the plot should be interpreted from right to left, and $\tau$ necessarily monotonically increases as we go back in time.
Without reionization, the optical depth steadily increases to the still small value $\tau \approx 10^{-1}$ at $z \approx -7$, just after the last recombination.
With reionization, this value is reached simply from reionization.
For $x \approx -7$, reionization no longer matters, and the optical depth quickly ramps up to $\tau \approx 10^{2}$, as the electrons become free.
The two earliest additional smaller bumps in $\tau''(x)$ are explained by the addition of Helium and the accompanying two additional recombination processes. \TODO{?}

\begin{figure}
	\centering
	\includegraphics[scale=0.7]{../plots/visibility_function.pdf}
	\caption{The visibility function $g(x)$ and its (scaled) derivatives with or without both Helium and reionization. The main plot best shows its variation during recombination, while the embedded plot magnifies its behavior during reionization. Their numerical integrals are annotated.}
	\label{fig_visibility_function}
\end{figure}

\Cref{fig_visibility_function} shows the visibility function.
It is a probability distribution, and does indeed numerically integrate very close to $1$.
Interpretation: probability that photon was last scattered at a given time $x$.
Greatest during recombination, $50$ times smaller at reionization, slowly declines till today.
Flat till today without reionization.

\clearpage
\section{Conclusions}

\TODO{}

\TODO{use article template}

%\bibliography{report.bib}
\printbibliography

\end{document}
